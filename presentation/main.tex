\documentclass[11pt,aspectratio=169]{beamer}

\usepackage[spanish]{babel}
\usepackage[utf8]{inputenc}
\usepackage[T1]{fontenc}
\usepackage{amsmath, amsfonts, amssymb}
\usepackage{graphicx}
\usepackage{booktabs}
\usepackage{hyperref}

\usetheme{Madrid}
\setbeamertemplate{navigation symbols}{}
\setbeamertemplate{footline}[frame number]
\usefonttheme{professionalfonts}

\title{Aplicaciones del Análisis Topológico de Datos (TDA)}
\subtitle{Resumen del informe}
\author{
  Pablo García Molina \and
  Antonio Lorenzo Díaz-Meco \and \\
  Andrés Martínez Fuentes \and
  Miguel Montes Lorenzo \and \\
  Francisco Javier Ríos Montes
}
\date{2025}

\begin{document}

\begin{frame}
  \titlepage
\end{frame}
% -----------------------------------------------------------
% INTRODUCCIÓN
% -----------------------------------------------------------

\begin{frame}{Introducción: Complejos simpliciales}

  Los complejos simpliciales aproximan un espacio usando puntos, aristas, triángulos
  y sus generalizaciones (símplejos de dimensión $k$).
  \bigskip

  Complejos relevantes:
  \begin{itemize}
    \item Complejo de Vietoris-Rips: incluye un símplex si todas las distancias por pares
          son menores o iguales que $\delta$.
    \item Complejo de Čech: incluye un símplex si las bolas centradas en sus vértices
          tienen intersección común.
  \end{itemize}

  \vfill

  \begin{center}
    \begin{columns}[c]

      \begin{column}{0.48\textwidth}
        \begin{center}
          \includegraphics[width=0.4\linewidth]{figures/Vietoris-Rips-example}
          \par\smallskip
          {\footnotesize Complejo de Vietoris-Rips}
        \end{center}
      \end{column}

      \begin{column}{0.48\textwidth}
        \begin{center}
          \includegraphics[width=0.6\linewidth]{figures/Cech-example}
          \par\smallskip
          {\footnotesize Complejo de Čech}
        \end{center}
      \end{column}

    \end{columns}
  \end{center}

\end{frame}


\begin{frame}{Introducción: Homología persistente (intuición)}
  \begin{columns}[T]  % [T] alinea ambas columnas arriba (más limpio en presentaciones)

    %----------------------%
    % COLUMNA IZQUIERDA   %
    %----------------------%
    \begin{column}{0.48\textwidth}
      La homología persistente estudia cómo cambian las características topológicas
      de un conjunto de datos al variar una escala.
      \bigskip

      Ideas clave:
      \begin{itemize}
        \item A cada punto se le asocia una bola de radio $r$.
        \item Al crecer $r$ aparecen y desaparecen componentes, ciclos y cavidades.
        \item Cada característica tiene nacimiento y muerte.
        \item Las características con vida larga representan estructura real.
      \end{itemize}
    \end{column}

    %----------------------%
    % COLUMNA DERECHA     %
    %----------------------%
    \begin{column}{0.52\textwidth}
      \begin{center}
        \includegraphics[width=\linewidth]{figures/persistent_homology}
      \end{center}
    \end{column}

  \end{columns}
\end{frame}

% -----------------------------------------------------------
% ECOSISTEMA PYTHON
% -----------------------------------------------------------


\begin{frame}{Ecosistema Python: GUDHI}
  \begin{itemize}
    \item Biblioteca más completa para TDA.
    \item Construye múltiples complejos: Čech, Rips, Alpha, Witness.
    \item Usa \textit{simplex tree} para eficiencia.
    \item Incluye cálculo completo de homología persistente y diversas representaciones topológicas para análisis y visualización.
  \end{itemize}
  \centering\includegraphics[width=0.50\linewidth]{figures/Witness_complex_representation.png}
\end{frame}

\begin{frame}{Ecosistema Python: Ripser y Giotto--TDA}
  \textbf{Ripser}
  \begin{itemize}
    \item Ultra-rápido y eficiente en memoria.
    \item Basado en cohomología persistente.
    \item Limitado a complejos Vietoris--Rips.
  \end{itemize}
  \vspace{0.4cm}
  \textbf{Giotto--TDA}
  \begin{itemize}
    \item Diseñado para machine learning.
    \item Basado en la cohomología persistente y permite trabajar con grafos, series temporales, nubes de puntos, imágenes ...
    \item Integración directa con \texttt{scikit-learn}.
    \item Usa Ripser internamente.
  \end{itemize}
\end{frame}

\begin{frame}{Ecosistema Python: Comparativa}
  \centering
  \begin{tabular}{@{}llll@{}}
    \toprule
    \textbf{Característica} & \textbf{GUDHI} & \textbf{Ripser} & \textbf{Giotto} \\
    \midrule
    Objetivo                & Completo       & Velocidad       & ML              \\
    Complejos               & Muchos         & Rips            & No expone       \\
    Velocidad               & Media          & Muy alta        & Alta            \\
    Control                 & Muy alto       & Bajo            & Medio           \\
    ML                      & Limitado       & Limitado        & Excelente       \\
    \bottomrule
  \end{tabular}
\end{frame}

% -----------------------------------------------------------
% APLICACIONES
% -----------------------------------------------------------

\begin{frame}{Aplicaciones de homología persistente: \\Variedad del artefacto en el espacio de datos}


\end{frame}

\begin{frame}{Aplicaciones de homología persistente: neurociencia}
  \begin{columns}[T]

    %------------------%
    % COLUMNA IZQUIERDA
    %------------------%
    \begin{column}{0.48\textwidth}
      La homología persistente permite estudiar la organización global de las redes cerebrales.

      \bigskip

      \begin{itemize}
        \item Se construyen redes a partir de fMRI o EEG.
        \item Se filtra el grafo variando un umbral de conexión.
        \item Se comparan patrones topológicos entre sujetos sanos y enfermos.
      \end{itemize}
    \end{column}

    %------------------%
    % COLUMNA DERECHA
    %------------------%
    \begin{column}{0.52\textwidth}
      \begin{center}
        \includegraphics[width=\linewidth]{figures/brain}
      \end{center}
    \end{column}

  \end{columns}
\end{frame}


\begin{frame}{Aplicaciones de homología persistente: finanzas}

  En finanzas, el TDA ayuda a detectar cambios de régimen y burbujas de mercado.

  \bigskip

  \begin{itemize}
    \item Se usan \emph{embeddings} de retraso para representar la dinámica en $\mathbb{R}^d$.
    \item La homología persistente identifica ciclos robustos antes de colapsos.
    \item Cambios en los números de Betti pueden anticipar fases de alta volatilidad.
  \end{itemize}

  \vfill

  \begin{center}
    \includegraphics[width=0.9\textwidth]{figures/finance_bubble}
  \end{center}

\end{frame}


\begin{frame}{Aplicaciones de homología persistente: propagación epidémica}

  El TDA captura la evolución espacio-temporal de epidemias como la COVID-19.

  \bigskip

  \begin{itemize}
    \item Se representan regiones mediante puntos en alta dimensión (posición, tiempo, casos).
    \item Mapper y homología persistente revelan \emph{hotspots} y trayectorias típicas.
    \item Permite comparar patrones de propagación entre distintas zonas.
  \end{itemize}

  \vfill

  \begin{center}
    \includegraphics[width=0.6\textwidth]{figures/covid}
  \end{center}

\end{frame}


% -----------------------------------------------------------
% IMPLEMENTACIÓN PROPIA
% -----------------------------------------------------------

\begin{frame}{Implementación propia: presentación del problema}
  \textbf{Objetivo:} comparar la estructura topológica del latido cardíaco en:
  \begin{itemize}
    \item \textbf{ECG normal} — dataset NSRDB (PhysioNet).
    \item \textbf{ECG arrítmico} — MIT-BIH Arrhythmia Database.
  \end{itemize}

  \vspace{0.35cm}

  \textbf{Datos utilizados:}
  \begin{itemize}
    \item Segmentos reales de \textbf{30 segundos} por paciente.
    \item Remuestreo a la misma frecuencia para hacerlos comparables.
    \item Señales de una sola derivación (canal 0).
  \end{itemize}

  \vspace{0.3cm}
  \centering
  \includegraphics[width=0.45\linewidth]{figures/normal_timeseries.png}
  \includegraphics[width=0.45\linewidth]{figures/arritmico_timeseries.png}
\end{frame}

\begin{frame}{Implementación propia: instrumentalización de TDA en ECG}
  \textbf{Idea clave: el corazón sano es casi periódico.}

  \begin{itemize}
    \item En el \textbf{ECG normal}, los intervalos R--R son regulares → dinámica estable.
    \item En el \textbf{ECG arrítmico}, la periodicidad se rompe → dinámica irregular.
  \end{itemize}

  \vspace{0.25cm}

  \textbf{¿Por qué TDA?}
  \begin{itemize}
    \item La periodicidad genera un \textbf{ciclo topológico} robusto en el espacio de estados.
    \item Las arritmias deforman o destruyen ese ciclo.
    \item La homología persistente permite medir esta diferencia de forma geométrica.
  \end{itemize}

  \vspace{0.3cm}
  \centering
  \includegraphics[width=0.42\linewidth]{figures/normal_timeseries.png}
  \includegraphics[width=0.42\linewidth]{figures/arritmico_timeseries.png}
\end{frame}

\begin{frame}{Implementación propia: metodología (I)}
  \textbf{1. Embedding mediante ventana deslizante}
  \[
    e(t) = (x_t, x_{t+\tau}, \dots, x_{t+(d-1)\tau})
  \]

  \textbf{2. Construcción del complejo Vietoris-Rips}

  \vspace{0.2cm}

  \begin{itemize}
    \item Se conecta cada par de puntos cuya distancia es menor que un umbral $r$.
    \item A medida que $r$ crece, aparecen componentes, ciclos y cavidades.
  \end{itemize}
\end{frame}

\begin{frame}{Implementación propia: metodología (II)}
  \textbf{3. Cálculo de diagramas de persistencia}

  \vspace{0.3cm}
  \centering
  \includegraphics[width=0.55\linewidth]{figures/normal_persistence.png}
  \includegraphics[width=0.55\linewidth]{figures/arritmico_persistence.png}

\end{frame}

\begin{frame}{Implementación propia: metodología (III)}
  \textbf{4. PCA sobre el embedding}

  \centering
  \includegraphics[width=0.45\linewidth]{figures/normal_embedding.png}
  \includegraphics[width=0.45\linewidth]{figures/arritmico_embedding.png}


\end{frame}


\begin{frame}{Implementación propia: resultados del análisis}

  \textbf{ECG normal:}
  \begin{itemize}
    \item Ciclo altamente persistente en $H_1$.
    \item Señal periódica y estable.
  \end{itemize}

  \vspace{0.2cm}

  \textbf{ECG arrítmico:}
  \begin{itemize}
    \item Ausencia de ciclos persistentes.
    \item Geometría irregular $\Rightarrow$ pérdida de periodicidad.
  \end{itemize}

  \textbf{Interpretación fisiológica:}

  \begin{itemize}
    \item El ciclo persistente refleja el ritmo cardíaco regular.
    \item Las arritmias destruyen esta estructura topológica.
    \item La homología persistente capta robustamente la estabilidad del latido.
  \end{itemize}
\end{frame}

\section{Implementación propia: Perspectivas de clasificación}

 \begin{frame}{Perspectivas de clasificación}

   \textbf{Invariantes topológicos como descriptores:}

   \begin{itemize}
     \item Diagramas de persistencia
     \item Curvas de Betti
   \end{itemize}

   \textbf{Ventajas de usar TDA:}

   \begin{itemize}
     \item Robustos al ruido.
     \item Capturan geometría global de la señal.
     \item Reducen la dimensionalidad sin perder estructura.
   \end{itemize}

   \vspace{0.3cm}

   \centering
   \textbf{→ El TDA proporciona una guía geométrica esencial para el aprendizaje profundo.}

 \end{frame}

 % -----------------------------------------------------------
 % CIERRE
 % -----------------------------------------------------------

 \begin{frame}{Recapitulación final}
   \textbf{Ideas principales} de la presentación:
   \begin{itemize}
     \item El TDA permite encontrar ``agujeros'' en variedades de datos de alta dimensión.
     \item Al trabajar con datos reales, generalmente no-lineales, ofrece una ventaja comparativa respecto de otros tipos de análisis más básicos.
     \item En el caso de los ECG:
           \begin{itemize}
             \item un ECG normal genera una órbita casi periódica en el espacio de fases,
                   lo que produce \textbf{un ciclo dominante y muy persistente} en $H_1$;
             \item un ECG arrítmico rompe esta periodicidad, dando lugar a
                   \textbf{ciclos cortos y ruidosos} sin una estructura clara.
           \end{itemize}
     \item El ecosistema Python cuenta con herramientas (GUDHI, Ripser, Giotto-TDA) que permiten llevar a cabo este tipo de análisis de forma sencilla.
   \end{itemize}
 \end{frame}

\end{document}
